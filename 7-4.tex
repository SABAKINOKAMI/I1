\documentclass[12pt,a4j]{jarticle}
\title{実習7.1}
\author{15515031 小泉 宗之}
\date{2017年6月6日}

\begin{document}
\maketitle
$z=x*yi$を複素数とする。領域$D$は次の不等式で定義されているものとする。
\begin{center}
		$D = \{z = x + yi$ \smallskip $|$ \smallskip $|z| \leq 1, |z-1|\geq 1,|z+1|\geq 1, y>0 \}.$
\end{center}
$D$の頂点を求めたい。
\\
$|z|=\sqrt{x^2+y^2}$なので、$D$の境界は
\begin{center}
\begin{equation}
x^2+y^2 =1\label{eq:xyint}
=1
=1
\end{equation}
\end{center}
で与えられる。式(\ref{eq:xyint})からを引くことにより、
\begin{center}
\[
2x-1=0
\]
\end{center}
を得る。すなわち$x=\frac{1}{2}$となる。これを(\ref{eq:xyint})に代入すると
\begin{center}
\[
(\frac{1}{2})^2+y^2=\frac{1}{4}+y^2=1
\]
\end{center}
すなわち、$y^2=\frac{3}{4}$となる。$y>0$なので、これを式(\ref{eq:xyint})に代入して、$(x,y)=$となる。
以上より、図1.1で与えられる$D$の各頂点は$z=,,,$の三点で与えられる。
\end{document}
