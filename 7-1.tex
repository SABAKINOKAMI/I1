\documentclass[12pt,a4j]{jarticle}
\title{実習7.1}
\author{15515031 小泉 宗之}
\date{2017年6月6日}

\begin{document}
\maketitle
関数$f(x)$の定義を
\begin{equation}
			f(x)=x^{3} + {\beta}x^{2} - x^{2} + {\beta}x ({\beta}>1)\label{eq:xyint}
\end{equation}
とする. 関数$f(x)$の微分$df^{2}/dx$は次式のとおり
\begin{equation}
			\frac{df^{2}}{dx} = 3x^2 -2{\beta}x - 2x + \beta \label{eq:xyrint}
\end{equation}
式(2)が0になるときに極値をとる. このときの$x$は
\begin{equation}
			x = \frac{ {\beta} + 1 {\pm} \sqrt{ {\beta}^2 - {\beta}+1 } }  {3}\label{eq:xyrrint}
\end{equation}
\begin{table}[htb]
 \begin{tabular}{|c|c|c|c|c|c|} \hline

$x$ & ... & $\frac{ {\beta} + 1 - \sqrt{ {\beta}^2 - {\beta}+1 } }  {3} $ & ...&$\frac{ {\beta} + 1 + \sqrt{ {\beta}^2 - {\beta}+1 } }  {3}$ & ... \\ \hline
$f'(x)$ & + & 0 & - & 0 & + \\ \hline 
$f(x)$ & $\nearrow$ & 極大 & $\searrow$ & 極小 & $\nearrow$ \\ \hline
 \end{tabular}
\end{table}
\end{document}
