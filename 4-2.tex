\documentclass[12pt,a4j]{jarticle}
\title{実習4.2}
\author{15515031 小泉 宗之}
\date{2017年5月30日}

\begin{document}
\maketitle

次はCのサンプルである。
\begin{verbatim}
#include <stdio.h>
int main()
{
  int i=2;
    printf("%d %d\n", i, i*i) ;

  return 0;
}
\end{verbatim}
ここで、\verb|#include| はプリプロセッサへの命令で... また、\verb+%d+で整数型の変数を表す.

上杉家の名君として次の二人が有名。
\begin{description}
\item[上杉謙信] 1530 -- 1578没。
米沢上杉家の藩祖。名将と謳われる。
\item[上杉鷹山] 1751 -- 1822没。
第9代米沢藩主。 中興の名君。
\end{description}

the latest CD by ''PUR'' consists of 4 songs with the title:
\begin{enumerate}
\item Ouvert\"ure
\item Hold me tight
\item If you hear this Tango
\item Adventure land.
\end{enumerate}

山形大学米沢キャンパス組織図 :
\begin{itemize}
\item 工学部 (6学科)
 \begin{itemize}
 \item 情報科学科
 \item 電気電子工学科
 \item 応用生命システム工学科
 \item (他3学科略)
 \end{itemize}
\item 大学院理工学研究科(9専攻)
 \begin{itemize}
 \item 情報科学専攻
 \item 電気電子工学専攻
 \item 応用生命システム工学専攻
 \item (他6専攻略)
 \end{itemize}
\end{itemize}

強調のため、ある単語または文を
中心に置くには
\begin{center}
center 環境
\end{center}
を用いる。

\begin{center}
\begin{tabular}{r|lc}
\hline
国名 & 首都 & 通貨 \\
\hline
日本 & 東京 & 円 \\
イギリス & ロンドン & ポンド \\
アルゼンチン&ブエノスアイレス&ペソ\\
\hline
\end{tabular}
\end{center}

\end{document}
